\documentclass[12pt]{fefu}

\setstretch{1}
\setlength{\parindent}{1cm}

\newcommand{\pa}[1]{\left(#1\right)}
\makeatletter
\renewcommand\@biblabel[1]{#1.}
\makeatother

\begin{document}
    \thispagestyle{empty}
    \newgeometry{left=2cm,right=2cm,top=2cm,bottom=2cm}
    \begin{center}
        \textbf{КОМПЛЕКС ПРОГРАММ ДЛЯ МОДЕЛИРОВАНИЯ ТРЁХМЕРНЫХ ЗВУКОВЫХ ПОЛЕЙ В СЛОЖНЫХ НЕОДНОРОДНЫХ ОКЕАНИЧЕСКИХ ВОЛНОВОДАХ}\\
        \vskip 12pt
        \textit{Тыщенко А. Г.}\\
        \vskip 12pt
        Дальневосточный федеральный университет (ДВФУ, Россия)\\
        \vskip 12pt
        \underline{ggoldenfeniks@gmail.com}
    \end{center}
    \par Моделирование трёхмерных звуковых полей применяется во многих областях исследования и освоения океана, например подводная акустическая навигация \cite{navigation19,navigation20}, оценка и минимизация влияния антропогенных шумов, создаваемых при добыче нефти, газа и различных биоресурсов, которые могут негативно влиять на морскую фауну. 
    \par Звуковое поле $p\pa{x,y,z}$ (где $z$ обозначает глубину, а  $x,y$ \textemdash координаты горизонтальной плоскости), создаваемое точечным источником в трёхмерном волноводе мелкого моря, расположенным по координатам $x=y=0$, $z=z_s$, описывается трёхмерным уравнением Гельмгольца \cite{jensen}
    \begin{equation}\label{eq::3DH}
        \pa{\rho\pa{x,y,z}\nabla\cdot\pa{\frac{1}{\rho\pa{x,y,z}}\nabla} + K\pa{x,y,z}^2}p\pa{x,y,z}=-\delta\pa{x}\delta\pa{y}\delta\pa{z-z_s}
    \end{equation}
    где $\rho\pa{x,y,z}$  плотность среды. На текущий момент существует несколько программных продуктов позволяющих вычислять решение уравнения Гельмгольца \cite{bellhop,traceo}, основанные на суммировании Гауссовых пучков, лучевой теории распространения звука, трёхмерных параболических уравнениях. Однако все они обладают существенными недостатками: использование геометроакустического приближения, запредельные требования к памяти и невероятно низкая скорость вычислений. Альтернативой этим методам служит представление звукового в виде
    \begin{equation}
        p\pa{x,y,z}=A\pa{x,y}\varphi\pa{z}\,,
    \end{equation}
    где $A_j\pa{x,y}$ --- модовые амплитуды, являющиеся решениями уравнения горизонтальной рефракции
    \begin{equation}\label{eq::HRE}
        \pa{\Delta+k_j^2\pa{x,y}}A_j\pa{x,y}=-\varphi_j\pa{z_s}\delta\pa{x}\delta\pa{y}\,.
    \end{equation}
    Модовые функции $\varphi_j\pa{z,x,y}$ и соответствующие им горизонтальные волновые числа $k_j\pa{x,y}$ находятся из решения акустической спектральной задачи \cite{jensen}. Такое представление позволяет многократно ускорить вычисление звукового поя без существенной потери точности и области применения \cite{dd}.
    \par Моделирование трёхмерного звукового поля требует проведения огромного количества вычислений, учитывающих различные параметры сложных неоднородных океанических волноводов. На текущий момент все этапы моделирования проводит сам исследователь вручную: сбор данных, обработка, приведение к необходимому формату, формирование входных параметров среды, обработка выходных данных, визуализация. Такое количество ручных задач сильно замедляет процесс моделирования, поэтому возникает необходимость в создании системы, которая позволила бы автоматизировать рутинную работу специалиста.
    \vskip 12pt
    \begingroup
        \titlespacing{name=\section,numberless}{0pt}{0pt}{12pt}
        \bibliographystyle{ugost2008ls}
        \bibliography{../references.bib}
    \endgroup
\end{document}