\documentclass[10pt, unicode]{beamer}

\usepackage{fontspec}
\usepackage{polyglossia}
\setdefaultlanguage{russian}
\setotherlanguage{english}
\setsansfont{Fira Sans}

\usepackage{nicefrac}

\usepackage{lmodern}

\newif\ifmetropolis
\metropolistrue

\title[Моделирование звукового поля]{Комплекс программ для моделирования трёхмерных звуковых полей в сложных неоднородных океанических волноводах}
\author[Тыщенко А.Г.]{
    \vbox{\raggedright%
        Студент группы М9119-09.04.01иибд\\ 
        Тыщенко Андрей Геннадьевич%
    }
    \vskip 20pt%
    \indent\vbox{\raggedright%
        Руководитель:\\
        кандидат физико-математических наук,\\
        зав. лаб. 3/2 ТОИ ДВО РАН
        Петров Павел Сергеевич\ifmetropolis\else\vskip -0.5cm\fi%
    }
}
\date{\today}

\newif\ifPS
%\PStrue

\ifmetropolis
    \usetheme[progressbar=frametitle,numbering=fraction]{metropolis}
    \makeatletter
    \setlength{\metropolis@progressinheadfoot@linewidth}{2pt}
    \makeatother
\else
    \usetheme{CambridgeUS}
\fi

\setbeamersize{text margin left=3.5mm,text margin right=3.5mm} 

\newcommand{\pa}[1]{\left(#1\right)}

\begin{document}
    
    \frame{\thispagestyle{empty}\titlepage}
    
    \begin{frame}[fragile]{Звуковое поле}
        \begin{block}{Уравнение Гельмгольца}
            \ifmetropolis
                \smallskip
            \fi
            \begin{equation}\label{eq::3DH}
                \pa{\rho\pa{z}\nabla\cdot\pa{\frac{1}{\rho\pa{z}}\nabla} +  K^2\pa{x,y,z}}p\pa{x,y,z}=-\delta\pa{x}\delta\pa{y}\delta\pa{z-z_s}
            \end{equation}
            \begin{align*}
                K\pa{z}&=\frac{\omega}{c\pa{z}}&&K\pa{z}=\frac{\omega}{c\pa{z}}\pa{1+i\eta\beta}
            \end{align*}
            \centering
            \begin{tabular}{ll}
                 $\rho\pa{z}$ --- плотность среды & $c\pa{z}$ --- скорость звука\\
                 $\omega=2\pi f$ --- циклическая частота & $\beta$ --- коэффициент затухания\\
                 $\eta=\nicefrac{1}{40\pi\log_{10}e}$&
            \end{tabular}
        \end{block}
    \end{frame}

    \begin{frame}[fragile]{Модовое разложение поля}
        \begin{block}{Модовое разложение}
            \ifmetropolis
                \smallskip
            \fi
            \begin{equation}
                p\pa{x,y,z}=\sum\limits_{j=1}^JA_j\pa{x,y}\varphi_j\pa{z,x,y}
            \end{equation}
        \end{block}
        \begin{block}{Уравнение горизонтальной рефракции}
            \ifmetropolis
                \smallskip
            \fi
            \begin{equation}
                \frac{\partial^2 A_j}{\partial x^2} + \frac{\partial^2 A_j}{\partial y^2}+k_j^2 (x,y)A_j=-\varphi_j(z_s)\delta(x)\delta(y)
            \end{equation}
        \end{block}
        \begin{block}{Псевдодифференциальное МПУ}
            \ifmetropolis
                \smallskip
            \fi
            \begin{equation}
                A_j\pa{x,y}=e^{k_{j,0}x}\mathcal{A}_j\pa{x,y}
            \end{equation}
            \begin{equation}
                \frac{\partial\mathcal{A}_j\pa{x,y}}{\partial x}=ik_{j,0}\pa{\sqrt{1+L_j}-1}\mathcal{A}_j\pa{x,y}
            \end{equation}
            \begin{equation}
                k_{j,0}^2L_j=\frac{\partial^2}{\partial y^2}+k_j^2\pa{x,y}-k_{j,0}^2\nonumber
            \end{equation}
        \end{block}
    \end{frame}

    \begin{frame}[fragile]{Горизонтальные лучи распространения звука}
        \begin{block}{}
            \ifmetropolis
                \smallskip
            \fi
            \begin{equation}
                A_j\pa{x,y}=M_j\pa{x,y}e^{ik_{j,0}S_j\pa{x,y}}+o\pa{\nicefrac{1}{k_{j,0}}}
            \end{equation}
        \end{block}
        \begin{block}{Уравнение Гамильтона-Якоби}
            \ifmetropolis
                \smallskip
            \fi
            \begin{equation}
                \pa{\frac{\partial S_j\pa{x,y}}{\partial x}}^2+\pa{\frac{\partial S_j\pa{x,y}}{\partial y}}^2=n_j\pa{x,y}
            \end{equation}
            \begin{equation*}
                n_j\pa{x,y}\equiv \nicefrac{k_j\pa{x,y}}{k_{j,0}}
            \end{equation*}
        \end{block}
        \begin{block}{Гамильтонова система}
            \ifmetropolis
                \smallskip
            \fi
            \begin{equation}
                \begin{aligned}
                    \frac{dx_j\pa{l}}{dl}&=\frac{\xi_j\pa{l}}{n_j\pa{x,y}}\qquad&\frac{d\xi_j\pa{l}}{dl}&=\frac{\partial n_j\pa{x,y}}{\partial x}\\
                    \frac{dy_j\pa{l}}{dl}&=\frac{\eta_j\pa{l}}{n_j\pa{x,y}}\qquad&\frac{d\eta_j\pa{l}}{dl}&=\frac{\partial n_j\pa{x,y}}{\partial y}\\
                \end{aligned}
            \end{equation}
        \end{block}
    \end{frame}
    
    \begin{frame}[fragile]{Области применения моделирования звука}
        \begin{block}{}
            \begin{itemize}
                \item Оценка влияния антропогенных шумов на морскую фауну
                \item Акустическая навигация и связь
            \end{itemize}
        \end{block}
        \begin{block}{Проблемы}
            \begin{itemize}
                \item ADHOC решения обработки входных данных, расчёта и визуализации звукового поля
                \item Наличие большого количества форматов входных и выходных данных
                \item Отсутствие пользовательского интерфейса указания параметров эксперимента и визуализации его результатов
            \end{itemize}
        \end{block}
    \end{frame}
    
    \begin{frame}[fragile]{Цель работы}
        \begin{block}{}
            Разработка комплекса программ для моделирования трёхмерного звукового поля и лучей распространения звукового сигнала с использованием модового разложения звука, предоставляющего удобный пользовательский интерфейс для проведения вычислительного эксперимента
        \end{block}
        \begin{block}{Задачи}
            \begin{itemize}
                \item Реализовать численные методы решения уравнения горизонтальной рефракции с использованием аппроксимации Паде произвольного порядка
                \item Реализовать метод вычисления горизонтальных лучей распространения звука
                \item Разработать программу для автоматизированного проведения вычислительных экспериментов с возможностью визуализации результатов
            \end{itemize}
        \end{block}
    \end{frame}

    \begin{frame}[fragile]{Обзор существующих решений}
        \begin{block}{Решение уравнения Гельмгольца}
            \begin{itemize}
                \item BELLHOP
                \item Traceo3D
                \item Закрытые программные продукты Океанографического института в Вудс-Хоуле и центральной школы Лиона
            \end{itemize}
        \end{block}
        \begin{block}{Визуализация результатов}
            \begin{itemize}
                \item Gnuplot
                \item Matlab
                \item Wolfram Mathematica
            \end{itemize}
        \end{block}
    \end{frame}

    \begin{frame}
        \thispagestyle{empty}
        \addtocounter{framenumber}{-1}
        \ifPS
            \centerline{\Huge Спасибо за внимание}
        \fi
    \end{frame}
    
\end{document}