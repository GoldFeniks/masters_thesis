\documentclass{fefu}

\usepackage{nicefrac}

\newcounter{slide}
\setcounter{slide}{1}
\newcommand{\slide}{\vskip 1cm\par {\bf Слайд №\theslide}\par\stepcounter{slide}}

\setlength{\parindent}{0pt}
\setstretch{1}

\newcommand{\pa}[1]{\left(#1\right)}

\begin{document}
    \newgeometry{left=2cm,right=2cm,top=2cm,bottom=2cm}
    \slide Защищается студент группы М9119-09.04.01иибд Тыщенко Андрей Геннадьевич по теме Комплекс программ для моделирования трёхмерных звуковых полей в сложных неоднородных океанических волноводах. Руководитель кандидат физико-математических наук, заведующий лабораторией 3/2 ТОИ ДВО РАН Петров Павел Сергеевич.
    \slide Звуковое поле, создаваемое точечным источником в трёхмерном волноводе мелкого моря, описывается трёхмерным уравнением Гельмгольца. Здесь $x,y$ --- горизонтальные координаты, $z$ --- глубина (положительная величина), $\rho\pa{z}$ --- плотность. Может быть рассмотрено два случая: с учётом затухания волн и без. От этого зависит выбор вида коэффициента $K$. $c\pa{z}$ --- скорость звука на глубине $z$. $\omega$ --- циклическая частота ($f$ -- частота источника), $\beta$ --- коэффициент затухания, $\eta$ --- вот такое странное число.
    \slide Решение $p\pa{x,y,z}$ уравнения Гельмгольца может быть выражено в форме модового разложения, где $A_j$ --- модовые амплитуды, а $\varphi_j$ ---модовые функции. Модовые аплитуды удовлетворяю уравнению горизонтальной рефракции, где $z_s$ --- глубина источника. Модовые функции $\varphi_j\pa{z,x,y}$ и соответствующие им горизонтальные числа $k_j(x,y)$ могут быть получены из решения акустической спектральной задачи. Таким образом появляется возможность рассматривать уравнение для каждой моды отдельно. Учитывая звуковые волны, распространяющиеся в положительном направлении оси $x$, вводя относительное волновое число $k_{j,0}$ и исключая главную осцилляцию получим псевдо-дифференциальное модовое параболическое уравнение.
    \slide Предполагая, что волновые числа $k_j\pa{x,y}$ являются медленно изменяющейся функцией, решение уравнения с использованием лучевой теории распространения звука может быть выражено в виде, где функция $M_j\pa{x,y}$ --- амплитуда нулевого порядка, а функция $S_j\pa{x,y}$ называется эйконалом и может быть найдена из уравнения Гамильтона-Якоби, где $n_j\pa{x,y}\equiv \nicefrac{k_j\pa{x,y}}{k_{j,0}}$ --- индекс горизонтальной рефракции. Решение этого уравнения связано с решением так называемой Гамильновой системы, где $l$ является натуральным параметром, обозначающим длину кривой вдоль траектории распространения луча, а $\xi,\eta$ сопряжённые переменные к $x,y$ --- момент. Из решения этой системы определяются траектории горизонтальных лучей распространения звука.
    \slide Моделирование трёхмерных звуковых полей применяется во многих областях исследования и освоения океана. Наиболее важными являются оценка влияния человека на морскую фауну и системы акустической навигации связи. При решении первой задачи необходимо оценить влияние существующих источников звука (так как проведение замеров является дорогостоящей операцией) и определить влияние новых при планировании. При разработке систем подводной навигации важно знать зоны уверенного сигнала и зоны тени, которые будут образованы в зависимости от взаимного расположения источников звука. Также важной задачей для навигации является вычисление лучей распространения звука с целью определения степени их искривления, влияющей на задержку. Основными проблемами таких исследований является отсутствие единой системы моделирования: для каждого эксперимента процесс моделирования воспроизводится практически с нуля.
    \slide Таким образом, целью работы является разработка комплекса программ для моделирования трёхмерного звукового поля и лучей распространения звукового сигнала с использованием модового разложения звука, предоставляющего удобный пользовательский интерфейс для проведения вычислительного эксперимента. Для достижения этой задачи были поставлены следующие цели.
    \slide На данный момент существует несколько программных продуктов позволяющих вычислять численное решение уравнения Гельмгольца. BELLHOP и \\Traceo3D, основанные на методе суммирования Гауссовых пучков и лучевой теории распространения звука соответственно. Недостатком этих методов является использование геометроакустического приближения, которое является недостаточно точным при моделировании источников звука, имеющих частоту менее 1 кГц. Океанографический институт в Вудс-Хоуле и центральная школа Лиона имеют закрытые комплексы программ, основанные на решении трёхмерного параболического уравнения, однако решение таких уравнений требует запредельных затрат памяти и времени, расчёт самых простых задач занимает не менее суток. Для визуализации вычислений чаще всего используются специализированные пакеты,например Gnuplot, Matlab, Wolfram Mathematica. Такие программы предлагают широкие возможности для визуализации, однако усложняют процесс работы, добавляя в него дополнительный шаг.
\end{document}