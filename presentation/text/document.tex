\documentclass{fefu}
\usepackage{nicefrac}

\newcounter{slide}
\newcommand{\slide}{\stepcounter{slide}\par\noindent\textbf{Слайд №\theslide}\par\noindent}

\newcommand{\pa}[1]{\left(#1\right)}

\setstretch{1}

\begin{document}
    \slide
    Защищается студент группы М9119-09.04.01иибд Тыщенко Андрей Геннадьевич по теме Комплекс программ для моделирования трёхмерных звуковых полей в сложных неоднородных океанических волноводах. Руководитель кандидат физико-математических наук, заведующий лабораторией 3/2 ТОИ ДВО РАН Петров Павел Сергеевич.
    \slide
    Моделирование трёхмерных звуковых полей применяется во многих областях исследования и освоения океана, например оценка влияния человека на морскую фауну и системы акустической навигации связи. При решении первой задачи необходимо оценить влияние источников звука на морскую фауну. При разработке систем подводной навигации важно знать зоны уверенного сигнала и зоны тени.
    \slide
    Целью работы является разработка комплекса программ для моделирования трёхмерного звукового поля с использованием модового разложения звука, предоставляющего возможность расчёта временного ряда импульса звукового сигнала и интегральных характеристик звукового сигнала c применением Split-step Pade метода решения уравнения горизонтальной рефракции и аппроксимации Паде произвольного порядка
    \slide
    Звуковое поле, создаваемое точечным источником в трёхмерном волноводе мелкого моря, описывается трёхмерным уравнением Гельмгольца. Здесь $x,y$ --- горизонтальные координаты, $z$ --- глубина (положительная величина), $\rho\pa{z}$ --- плотность. Может быть рассмотрено два случая: с учётом затухания волн и без, зависящее от значения коэффициента затухания $\beta$. Здесь $c\pa{z}$ --- скорость звука на глубине $z$. $\omega$ --- циклическая частота ($f$ -- частота источника), $\eta$ --- вот такое странное число.
    \slide
    Решение $p\pa{x,y,z}$ уравнения Гельмгольца может быть выражено в форме модового разложения, где $A_j$ --- модовые амплитуды, а $\varphi_j$ ---модовые функции. Модовые аплитуды удовлетворяют уравнению горизонтальной рефракции, где $z_s$ --- глубина источника. Модовые функции $\varphi_j\pa{z,x,y}$ и соответствующие им горизонтальные числа $k_j(x,y)$ могут быть получены из решения акустической спектральной задачи. Таким образом появляется возможность рассматривать уравнение для каждой моды отдельно. Учитывая звуковые волны, распространяющиеся в положительном направлении оси $x$ и вводя относительное волновое число $k_{j,0}$ получим псевдо-дифференциальное модовое параболическое уравнение.
    \slide
    Для решения псевдо-дифференциального модового параболического уравнения необходимо выполнить линеаризацию оператора квадратного корня. Для этого может быть использована аппроксимация Паде. Порядок этой аппроксимации определяет широкоугольные свойства решения. С использованием метода Крэнка-Николсон псевдодифференциальное МПУ может быть приведено к следующему виду. Далее дискретизация оператора $L_j^\delta$ производится с использованием конечных разностей
    \slide
    Также существует другой подход, изначально предложенный Коллинзом. В его основе лежит смена порядка дискретизации и применения аппроксимации Падэ. При достаточно малом шаге $h$ решение псевдодифференциального МПУ может быть выражено в виде пропогатора по $x$. Затем аппроксимация Падэ применяется к экспоненте. Полученное уравнение совпадает с дискретизацией методом Крэнка-Николсон с точностью до значений коэффициентов, и дискретизация оператора $L_j^\delta$ может быть также выполнена конечными разностями
    \slide
    Одной из особенностей МПУ является то, что их решение всегда рассмат­ривается в неограниченной области, поэтому её искусственное ограничение является обязательным при численном решении МПУ. В данной работе были использованы PML граничные условия, которые впервые были использованы Беренджером для уравнений Максвелла. В основе метода лежит расширение вычислительной области с целью плавного поглощения волн исходящих из неё, для этого оператор $L_j$ заменяется на оператор $L_j^{PML}$, где $\sigma\pa{y}$ монотонная функция, возрастающая при движении вглубь PML слоёв и равная нулю в области $\Omega$. В рамках данной работы была использована кубическая функция, где $y_b$ --- соответствующая граница области, а $\sigma_0$ максимальное значение функции. При этом на новых границах ставятся обычные нулевые условия Дирихле
    \slide
    При решении очень широкоугольных уравнений требуются соответствующие широкоугольные начальные условия, чтобы избежать численного шума, так как наиболее часто используемые начальные условия Гаусса и Грина создают численный шум даже при использовании первого члена аппроксимации. Для борьбы с этим могут быть использованы лучевые начальные условия, рассматриваемые на расстоянии $x_0$ от источника, сравнимым с длиной волны. Начальное условие вычисляется с использованием лучевой теории звука, при этом предполагается, что среда не изменяется с координатой $x$. Так как $x_0$ обычно достаточно мало, можно в дальнейшем также предположить независимость от $y$, что приводит к упрошенным начальным условиям. 
    \slide
    Также может быть рассмотрена задача вычисления временного ряда импульса звукового сигнала в произвольных точках среды. Пусть источник излучает сигнал $g\pa{t}$. Тогда импульс $I$ в приёмнике может быть вычислен в спектральной (частотной) области. Здесь $p$ это решение уравнения Гельмгольца для частоты $f$, $\tau$ --- время, которое требуется звуку для достижения приёмника. Также может быть вычислен Sound Exposure Level, который является интегралом модуля квадрата звукового давления при диапазоне рассматриваемых частот $f_1, f_2$ и обозначает уровень шума в данной точке
    \slide
    На данный момент существует несколько программных продуктов позволяющих вычислять численное решение уравнения Гельмгольца. BELLHOP и Traceo3D основанны на лучевой теории распространения звука. Океанографический институт в Вудс-Хоуле и центральная школа Лиона имеют закрытые комплексы программ, основанные на решении трёхмерного параболического уравнения. Недостатком этих методов являются большие затраты памяти и времени, расчёт самых простых задач занимает не менее суток.
    \slide
    Основным входным файлом программы является конфигурационный файл, который позволяет задавать параметры среды, источника и приёмников. Выходными данными являются несколько файлов в формате JSON, описывающие информацию о проведённых вычислениях, использованных параметрах, а также описание многомерных данных, полученных в результате работы. Этим могут является звуковое поле источника, начальные условия, модовые функции и волновые числа, импульс звукового сигнала, SEL, координаты распространения лучей
    \slide
    В качестве первого эксперимента было проведено моделирование распространения звука в волноводе с постоянной глубиной дна. Полученное решение сравнивалось с аналитическим решением и решением ШМПУ с использованием аппроксимации Клаербоута квадратного корня и начальных условий Грина. Как видно из рисунка использование SSP метода с простыми лучевыми начальными условиями и большим порядком аппроксимации позволяет получить решение, которое почти идеально аппроксимирует аналитическое
    \slide
    Также была изучена работа PML граничных условий, так, исходящие из вычислительной области волны постепенно затухают при движении вглубь поглощающего слоя
    \slide
    Далее был рассмотрен подводный каньон. Решение было получено с использованием 11 членов аппроксимации и простых лучевых начальных условий. Как видно из рисунка, полученное решение имеет апертуру почти $\pm 90^\circ$
    \slide
    Полученное решение также сравнивалось с решением трёхмерного параболического уравнения при $y=0$ на глубине $10$ метров. Из рисунка видно, что решения достаточно сильно совпадают, при этом решение МПУ требует значительно меньше вычислительных ресурсов
    \slide
    Также была проведена трассировка лучей, соответствующих вертикальным модам. Каньон захватывает звук, поэтому лучи образуют петли при достаточно небольшом угле отклонения от главной оси распространения. С увеличением номера моды захват становится более выраженным
    \slide
    Далее было рассмотрено моделирование распространения звука в клиновидном волноводе.
    \slide
    Было произведено сравнения акустического поля на различных горизонтах $x$. Из рисунка видно, что вблизи источника SSP решение не образует численного шума, а при отдалении от источника становится заметна более широкая апертура этого решения.
    \slide
    Также было произведено сравнение с решением методом изображений. Решения всех методов почти совпадают, не смотря на адиабатическую природу модовых параболических уравнений, при этом большая апертура SSP метода сильнее приближает решение к решению методом изображений вдали от источника. Время вычисления решения методом изображений составляет приблизительно 23 часа, по сравнению с несколькими минутами, затрачиваемыми на решение модового параболического уравнения
    \slide
    Также была рассмотрена задача с настоящими данными батиметрии и аппроксимацией профиля скорости звука в воде
    \slide
    Звуковое поле было вычислено на глубине $z=4\ \text{м.}$ Из рисунка видно, что решение ШМПУ существенно уступает решению, полученному методом SSP с большим порядком аппроксимации Падэ. Также можно заметить, как звук фокусируется в области с большей глубиной.
    \slide
    Для выполнения поставленных задач была написана программа на языке С++17. Репозиторий с кодом размешен на гитхабе.
    \slide
    Таким образом, в рамках проделанной работы были исследованы методы решения МПУ с использованием аппроксимации Падэ произвольного порядка, лучевых начальных условий и PML граничных условий, разработана численная схема их решения, разработан комплекс программ на языке C++, реализующий полученную численную схему с использованием пакета CAMBALA и возможностью вычисления временного ряда импульса звукового сигнала, значения распределения уровней SEL и координат распространения лучей, соответствующих вертикальным модам; проведены различные вычислительные эксперименты и изучена корректность и применимость полученного метода в сравнении с ШМПУ и другими методами решения уравнения Гельмгольца.
    \slide
    Спасибо за внимание
\end{document}\n
