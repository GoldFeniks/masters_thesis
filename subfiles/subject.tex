\documentclass[../document.tex]{subfiles}

\begin{document}
	\section{Описание предметной области}
		\subsection{Оценка влияния антропогенных шумов}
			\par Моделирование трёхмерных звуковых полей применяется во многих областях исследования и освоения океана, требующих учёта множества параметров сложных неоднородных океанических волноводов.
			\par С развитием промышленности все больше расширяется хозяйственная деятельность человека, связанная с добычей нефти, газа и разнообразных биоресурсов в акватории мирового океана, в результате которой создаётся огромное количество антропогенных шумов, которые негативно сказываются на морской фауне \cite{noise1, noise2}. Таким образом, возникает задача оценивания и минимизации шумового загрязнения и его воздействия на мировой океан. Существуют два подхода к решению этой задачи: проведение измерений и моделирование. В первом случае проводится некоторое количество замеров плотности звука в воде, по которым строится интерполяционная картина звукового поля. Недостатком такого метода является дороговизна и сложность проведения измерений, которые также фактически могут быть получены только в точках среды на сетке с очень большими шагами по координатам, поэтому такие данные чаще всего используются для корректировки и проверки точности модельных данных. В свою очередь моделирование требует данных о положении и характеристиках источника звука, а также данных о свойствах среды: батиметрии, гидрологии и структуре дна. Основным преимуществом моделирования является возможность вычисления звукового поля как и уже существующих источников, так и планируемых, что позволяет заранее минимизировать влияние человека на океан. Недостатком такого метода является необходимость сбора и обработки изменяющихся данных о состоянии среды, что само по себе является сложной задачей.
		\subsection{Акустическая навигация}
			\par С каждым годом хозяйственная деятельность человека всё больше осуществляется с использованием автономных подводных аппаратов, требующих наличия стабильных систем навигации и связи, основанных на распространении звука, так как электромагнитное излучение неприменимо в подобных условиях \cite{navigation19,navigation20}. При разработке систем подводной акустической навигации возникает задача определения зон уверенного приёма и поиска взаимного расположения источников звукового сигнала таким образом, чтобы минимизировать зоны акустической тени. Также существует задача расчёта траекторий распространения звука на несущих частотах сигналов, с целью определения искривления по сравнению с геодезической на поверхности Земли для вычисления задержки звукового сигнала при осуществлении подводной навигации.
		\subsection{Модовое представление звукового поля точечного источника\label{sec::sound_modes}}
			\par Звуковое поле $p\pa{x,y,z}$ (где $z$ обозначает глубину, а  $x,y$ --- координаты горизонтальной плоскости), создаваемое точечным источником в трёхмерном волноводе мелкого моря, расположенным по координатам $x=y=0$, $z=z_s$ и имеющего частоту $f$, описывается трёхмерным уравнением Гельмгольца \cite{jensen}
			\begin{equation}\label{eq::3DH}
				\pa{\rho\pa{x,y,z}\nabla\cdot\pa{\frac{1}{\rho\pa{x,y,z}}\nabla} + K\pa{x,y,z}^2}p\pa{x,y,z}=-\delta\pa{x}\delta\pa{y}\delta\pa{z-z_s}
			\end{equation}
			где $\rho\pa{x,y,z}$ --- плотность среды. Коэффициент $K\pa{x,y,z}$ определяется двумя способами:
			\begin{enumerate}
				\item Без учёта затухания:
				\begin{equation}\label{eq::K1}
					K\pa{x,y,z}=\frac{\omega}{c\pa{x,y,z}}\,,
				\end{equation}
				\item С учётом затухания:
				\begin{equation}\label{eq::K2}
					K\pa{x,y,z}=\frac{\omega}{c\pa{x,y,z}}\pa{1+i\eta\beta}\,,
				\end{equation}
			\end{enumerate}
			где $c\pa{x,y,z}$ --- скорость звука, $\omega=2\pi f$ --- циклическая частота, $\beta$ --- коэффициент затухания, $\eta=\frac{1}{40\pi\log_{10}e}$. Используя технику разделения переменных и представляя звуковое поле в виде
            \begin{equation}
                p\pa{x,y,z}=A\pa{x,y}\varphi\pa{z,x,y}\,,
            \end{equation}
            при $x^2+y^2>0$ и $z\ne z_s$ уравнение \eqref{eq::3DH} можно привести к виду
            \begin{multline}\label{eq::modal_pe}
                \frac{1}{A\pa{x,y}}\pa{\rho\pa{x,y,z}\nabla\cdot\pa{\frac{1}{\rho\pa{x,y,z}}\nabla}A\pa{x,y}}+\\
                \frac{1}{\varphi\pa{z,x,y}}\pa{\rho\pa{x,y,z}\frac{\partial}{\partial z}\pa{\frac{1}{\rho\pa{x,y,z}}\frac{\partial\varphi\pa{z,x,y}}{\partial z}}+K\pa{x,y,z}^2\varphi\pa{z,x,y}}=0\,.
            \end{multline}
            Вводя константу разложения $k_j^2$, из \eqref{eq::modal_pe} можно получить модовое уравнение
            \begin{multline}
                \rho\pa{x,y,z}\frac{\partial}{\partial z}\pa{\frac{1}{\rho\pa{x,y,z}}\frac{\partial\varphi_j\pa{z,x,y}}{\partial z}}+\\
                \pa{K\pa{x,y,z}^2+k_j^2\pa{x,y}}\varphi_j\pa{z,x,y}=0\,,
            \end{multline}
            с граничными условиями на $z=0$ и $z=z_b$ ($z_b$ -- глубина дна). Функция $\varphi_j\pa{z,x,y}$ называется модовой функцией, а $k_j\pa{x,y}$ соответствующим ей горизонтальным волновым числом. Используя граничные условия для исходного уравнения Гельмгольца, можно выписать задачу Штурма-Лиувилля, обладающую следующими свойствами при фиксированных $x,y$
            \begin{enumerate}
                \item задача имеет счётное количество решений -- мод,
                \item каждая такая мода характеризуется собственной функцией $\varphi_j\pa{z,x,y}$ и волновым числом $k_j\pa{x,y}$,
                \item без учёта затухания все волновые числа являются вещественными, с учётом -- комплексными,
                \item модовые функции образуют полную систему.
            \end{enumerate}
            Также, модовые функции и соответствующие им волновые числа выбраны таким образом, что
            \begin{enumerate}
                \item функции $\varphi_j\pa{z,x,y}$ имеют ровно $j$ корней на отрезке $\left[0,z_b\right]$,
                \item имеет место равенство
                \begin{equation}
                    k_1^2\pa{x,y}>k_2^2\pa{x,y}>k_3^2\pa{x,y}>\dots\,,
                \end{equation}
                \item модовые функции ортогональны и нормированы
                \begin{equation}
                    \int\limits_0^{z_b}\frac{\varphi_i\pa{z,x,y}\varphi_j\pa{z,x,y}}{\rho\pa{x,y,z}}\operatorname{dz}=\delta_{i,j}\,.
                \end{equation}
            \end{enumerate}
            Таким образом, звуковое поле $p\pa{x,y,z}$ может быть представлено в виде модового разложения
			\begin{equation}\label{eq::MD}
				p\pa{x,y,z}=\sum\limits_{j=1}^\infty A_j\pa{x,y}\varphi_j\pa{z,x,y}\,,
			\end{equation}
			где $A_j\pa{x,y}$ --- модовые амплитуды, являющиеся решениями так называемого уравнения горизонтальной рефракции
			\begin{equation}\label{eq::HRE}
				\pa{\Delta+k_j^2\pa{x,y}}A_j\pa{x,y}=-\varphi_j\pa{z_s}\delta\pa{x}\delta\pa{y}\,,
			\end{equation}
            которое может быть получено из уравнения Гельмгольца \eqref{eq::3DH} с использованием \eqref{eq::MD} и адиабатического приближения, заключающегося в пренебрежении межмодовым взаимодействием \cite{jensen}.
		\subsection{Модовые параболические уравнения}
			\par Для получения параболических аппроксимаций уравнение горизонтальной рефракции \eqref{eq::HRE} представляется в виде
			\begin{equation}
				\pa{\frac{\partial}{\partial x}+i\sqrt{k_j^2\pa{x,y}+\frac{\partial^2}{\partial y^2}}}\pa{\frac{\partial}{\partial x}-i\sqrt{k_j^2\pa{x,y}+\frac{\partial^2}{\partial y^2}}}A_j\pa{x,y}=0\,,
			\end{equation}
			и выделяется его решение, состоящее из волн, распространяющихся в положительном направлении оси $x$
			\begin{equation}
				\pa{\frac{\partial}{\partial x}-i\sqrt{k_j^2\pa{x,y}+\frac{\partial^2}{\partial y^2}}}A_j\pa{x,y}=0\,.
			\end{equation}
			Вводя модовое опорное волновое число $k_{j,0}$ и выделяя главную осцилляцию из $A_j\pa{x,y}$
			\begin{equation}
				A_j\pa{x,y}=e^{k_{j,0}x}\mathcal{A}_j\pa{x,y}\,,\nonumber
			\end{equation}
			получим псевдодифференциальное модовое параболическое уравнение
			\begin{equation}\label{eq::PDMPE}
				\frac{\partial\mathcal{A}_j\pa{x,y}}{\partial x}=ik_{j,0}\pa{\sqrt{1+L_j}-1}\mathcal{A}_j\pa{x,y}\,,
			\end{equation}
			где 
			\begin{equation}
				k_{j,0}^2L_j=\frac{\partial^2}{\partial y^2}+k_j^2\pa{x,y}-k_{j,0}^2\,.\nonumber
			\end{equation}
		\subsection{Лучевая теория для уравнения горизонтальной рефракции}
			\par Предполагая, что волновые числа $k_j\pa{x,y}$ являются медленно изменяющейся функцией, решение уравнения \eqref{eq::HRE} с использованием лучевой теории распространения звука может быть выражено в виде
			\begin{equation}
				A_j\pa{x,y}=M_j\pa{x,y}e^{ik_{j,0}S_j\pa{x,y}}+o\pa{\nicefrac{1}{k_{j,0}}}\,,
			\end{equation}
			где функция $S_j\pa{x,y}$ называется эйконалом и может быть найдена из уравнения Гамильтона-Якоби
			\begin{equation}\label{eq::hamilton_jacoby}
				\pa{\frac{\partial S_j\pa{x,y}}{\partial x}}^2+\pa{\frac{\partial S_j\pa{x,y}}{\partial y}}^2=n_j\pa{x,y}\,,
			\end{equation}
			где $n_j\pa{x,y}\equiv \nicefrac{k_j\pa{x,y}}{k_{j,0}}$ --- индекс горизонтальной рефракции \cite{burridge}. Амплитуда $M_j\pa{x,y}$ может быть получена из уравнения переноса вида
            \begin{multline}\label{eq::transfer}
                2\pa{\frac{\partial S_j\pa{x,y}}{\partial x}\frac{\partial M_j\pa{x,y}}{\partial x}+\frac{\partial S_j\pa{x,y}}{\partial y}\frac{\partial M_j\pa{x,y}}{\partial y}}+\\\pa{\frac{\partial^2S_j\pa{x,y}}{\partial x^2}+\frac{\partial^2S_j\pa{x,y}}{\partial y^2}}M_j\pa{x,y}=0\,.
            \end{multline}
            Решение этих уравнений связано с решением так называемой Гамильновой системы
			\begin{equation}\label{eq:hamilton_system}
				\begin{aligned}
					\frac{dx_j\pa{l}}{dl}&=\frac{\xi_j\pa{l}}{n_j\pa{x_j,y_j}}\,,\qquad&\frac{d\xi_j\pa{l}}{dl}&=\frac{\partial n_j\pa{x_j,y_j}}{\partial x_j}\,,\\
					\frac{dy_j\pa{l}}{dl}&=\frac{\eta_j\pa{l}}{n_j\pa{x_j,y_j}}\,,\qquad&\frac{d\eta_j\pa{l}}{dl}&=\frac{\partial n_j\pa{x_j,y_j}}{\partial y_j}\,,\\
				\end{aligned}
			\end{equation}
			где $l$ является натуральным параметром, обозначающим длину кривой вдоль траектории распространения луча, а $\xi,\eta$ сопряжённые переменные к $x,y$ --- момент. Из решения этой системы определяются траектории горизонтальных лучей распространения звука.
		\subsection{Обзор существующих методов решения}
			\par На данный момент существует несколько программных продуктов позволяющих вычислять численное решение уравнения Гельмгольца \eqref{eq::3DH}. BELLHOP \cite{bellhop} и Traceo3D \cite{traceo}, основанные на методе суммирования Гауссовых пучков и лучевой теории распространения звука соответственно. Недостатком этих методов является использование геометроакустического приближения, которое является недостаточно точным при моделировании источников звука, имеющих частоту менее 1 кГц. Океанографический институт в Вудс-Хоуле \cite{whoi} и центральная школа Лиона \cite{lyon} имеют закрытые комплексы программ, основанные на решении трёхмерного параболического уравнения \cite{isakson14,lin12,shtrum16}, однако решение таких уравнений требует запредельных затрат памяти и времени, вычисление решения даже самых простых задач занимает не менее суток.
		\subsection{Неформальная постановка задачи}
			\par Таким образом, в рамках данной работы необходимо выполнить следующие задачи:
			\begin{enumerate}
				\item Исследовать методы моделирования трёхмерного звукового поля в сложных неоднородных океанических волноводах.
				\item Изучить способы решения \acrshort{mpe} с использованием метода Split-step Pad\'e и аппроксимации Паде произвольного порядка, а также \acrshort{pml} граничных условий.
				\item Разработать комплекс программ для моделирования трёхмерных звуковых полей с возможностью вычисления временного ряда импульса звукового сигнала, интегральной характеристики \acrshort{sel}, и координат распространения лучей, соответствующих вертикальным модам.
                \item Провести вычислительные эксперименты с целью оценки корректности и точности используемых методов.
			\end{enumerate}
			\subsubsection{Требования к программной реализации}
				\par На данный момент существует множество программных комплексов, сфокусированных на решении какой-то одной мелкой задачи, зачастую являющейся часть чего-то большего, или же направленных на решение какой-то одной конкретной задачи. Поэтому возникает необходимость в разработке новой программы, позволяющей решать некоторый спектр задач, не имея привязанности к определённым параметрам. Основными требованиями к разрабатываемой программе являются:
				\begin{enumerate}
					\item Реализация численных схем решения модовых параболических уравнений.
					\item Моделирование звукового поля на трёхмерной сетке.
					\item Возможность использования как готовых коэффициентов уравнения, так
					и коэффициентов, вычисленных с помощью пакета Cambala \cite{cambala}, с указанием необхо­димых параметров: плотности среды, батиметрии, гидрологии и др.
					\item Проведение трассировки горизонтальных лучей, соответствующих вертикальным модам.
					\item Вычисление временного ряда импульса звукового сигнала в произвольных точках среды.
					\item Оценка уровня шума с использованием интегральной характеристики \acrshort{sel}.
					\item Высокая скорость работы по сравнению с альтернативными методами моделирования.
				\end{enumerate}
\end{document}