\documentclass[../document.tex]{subfiles}

\begin{document}
	\begin{abstract}
		\par В работе рассматриваются различные задачи моделирования распространения звука в неоднородных океанических волноводах, в частности, трассировка лучей, соответствующих вертикальным модам, вычисление временного ряда импульсного акустического сигнала и SEL (Sound Exposure Level, уровень озвучивания). На текущий момент программные продукты, предназначенные для решения поставленных задач, являются очень разносортными, что существенно затрудняет процесс проведения исследований океана акустическими средствами. Появление комплекса программ, объединяющего в себе возможности для решения ряда задач моделирования распространения звука, позволит упростить работу исследователей, а также инженеров, выполняющих расчёты акустических полей при решении практических задач.
		\par Работа является логическим продолжением выпускной квалификационной работы бакалавра \cite{bachelor}. Рассмотрены широкоугольные модовые параболические уравнения, а также численные методы их решения. Проведён анализ применимости различных типов начальных условий. Добавлена поддержка вычисления временного ряда импульсного звукового сигнала, распределения уровней SEL, а также возможность расчёта указанных величин на нескольких горизонтах. Усовершенствован процесс ввода и вывода данных. Открытый исходный код программы расположен по адресу \href{https://github.com/GoldFeniks/Ample}{https://github.com/GoldFeniks/Ample}.
		\par {\it Ключевые слова: уравнение Гельмгольца, широкоугольные модовые параболические уравнения, импульс звукового сигнала, SEL, моделирование звукового поля.} 
	\end{abstract}
\end{document}