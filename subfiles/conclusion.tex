\documentclass[../document.tex]{subfiles}

\begin{document}
    \section*{Заключение}
    \par Таким образом в рамках проделанной работы были исследованы методы решения \acrshort{mpe} с использованием аппроксимации Паде произвольного порядка, лучевых начальных условий и \acrshort{pml} граничных условий, разработана численная схема их решения, разработан комплекс программ на языке C++, реализующий полученную численную схему с использованием пакета CAMBALA и возможностью вычисления временного ряда импульса звукового сигнала, значения распределения уровней \acrshort{sel} и координат распространения лучей, соответствующих вертикальным модам; проведены различные вычислительные эксперименты и изучена корректность и применимость полученного метода в сравнении с \acrshort{wampe} и другими методами решения уравнения Гельмгольца. В настоящее время комплекс программ уже используется сотрудниками лаборатории 2/4 ТОИ ДВО РАН для моделирования и оценки влияния антропогенных акустических шумов на шельфе о. Сахалин.
    \par На данный момент моделирование звукового поля основано на адиабатическом приближении, показывающем недостаточную точность при достаточно резком изменении батиметрии волновода, поэтому основным направлением дальнейшей работы является разработка численной схемы, учитывающей межмодовое взаимодействие, и её реализация в составе программного продукта. Также, разработанный комплекс программ не позволяет задавать детальную информацию о структуре слоёв воды и дна, что может существенно понизить точность модели в рамках некоторых задач. 
    \par Результаты работы были представлены на различных научных мероприятиях, включая PACON, UACE, Days on Diffraction \cite{dd}. Часть результатов работы была опубликована в \guillemotleft Journal of Sound and Vibration\guillemotright\ \cite{jsv}, а также направлена статья в \guillemotleft Акустический журнал\guillemotright\ \cite{acoustic_journal}, сфокусированная на разработанном программном продукте, в настоящий момент статья принята к печати и выйдет в конце 2021 г.
\end{document}