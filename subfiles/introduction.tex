\documentclass[../document.tex]{subfiles}

\begin{document}
    \subsection{Описание предметной области}
    \subsubsection{Оценка влияния антропогенных шумов}
        \par Моделирование трёхмерных звуковых полей применяется во многих областях исследования и освоения океана, требующих учёта множества параметров сложных неоднородных океанических волноводов.
        \par С развитием промышленности все больше расширяется хозяйственная деятельность человека, связанная с добычей нефти, газа и разнообразных биоресурсов в акватории мирового океана, в результате которой создаётся огромное количество антропогенных шумов, которые негативно сказываются на морской фауне \cite{noise1, noise2}. Таким образом, возникает задача оценивания и минимизации создаваемых шумов. Существуют два подхода к решению этой задачи: проведение замеров и моделирование. В первом случае проводится некоторое количество измерений плотности звука в воде, по которым строится интерполяционная картина звукового поля. Недостатком такого метода является дороговизна и сложность проведения замеров, поэтому такие данные чаще всего используются для корректировки и проверки точности модельных данных. В свою очередь моделирование требует данных о положении и характеристиках источника звука, а также данных о свойствах среды: батиметрии, гидрологии и структуре дна. Основным преимуществом моделирования является возможность вычисления звукового поля как и уже существующих источников, так и планируемых, что позволяет заранее минимизировать влияние человека на океан. Минусом такого метода является необходимость сбора и обработки изменяющихся данных о состоянии среды, что само по себе является сложной задачей.
    \subsubsection{Акустическая навигация}
        \par Не менее важным 
\end{document}