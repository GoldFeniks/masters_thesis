\documentclass[../document.tex]{subfiles}

\begin{document}
    \subsection{Описание предметной области}
        \subsubsection{Оценка влияния антропогенных шумов}
            \par Моделирование трёхмерных звуковых полей применяется во многих областях исследования и освоения океана, требующих учёта множества параметров сложных неоднородных океанических волноводов.
            \par С развитием промышленности все больше расширяется хозяйственная деятельность человека, связанная с добычей нефти, газа и разнообразных биоресурсов в акватории мирового океана, в результате которой создаётся огромное количество антропогенных шумов, которые негативно сказываются на морской фауне \cite{noise1, noise2}. Таким образом, возникает задача оценивания и минимизации шумового загрязнения и его воздействия на мировой океан. Существуют два подхода к решению этой задачи: проведение замеров и моделирование. В первом случае проводится некоторое количество измерений плотности звука в воде, по которым строится интерполяционная картина звукового поля. Недостатком такого метода является дороговизна и сложность проведения замеров, поэтому такие данные чаще всего используются для корректировки и проверки точности модельных данных. В свою очередь моделирование требует данных о положении и характеристиках источника звука, а также данных о свойствах среды: батиметрии, гидрологии и структуре дна. Основным преимуществом моделирования является возможность вычисления звукового поля как и уже существующих источников, так и планируемых, что позволяет заранее минимизировать влияние человека на океан. Минусом такого метода является необходимость сбора и обработки изменяющихся данных о состоянии среды, что само по себе является сложной задачей.
        \subsubsection{Акустическая навигация}
            \par С каждым годом хозяйственная деятельность человека всё больше осуществляется с использованием автономных подводных аппаратов, требующих наличия стабильных систем навигации и связи, основанных на распространении звука, так как электромагнитное излучение неприменимо в подобных условиях \cite{navigation19,navigation20}. При разработке систем подводной акустической навигации возникает задача определения зон уверенного приёма и поиска взаимного расположения источников звукового сигнала таким образом, чтобы минимизировать зоны акустической тени. Также существует задача расчёта траекторий распространения звука на несущих частотах сигналов от 300 до 500 Гц, с целью определения искривления звуковых лучей для вычисления задержки звукового сигнала при осуществлении подводной навигации.
        \subsubsection{Модовое представление звукового поля точечного источника}
            \par Звуковое поле $p\pa{x,y,z}$ (где $z$ обозначает глубину, а  $x,y$ --- координаты горизонтальной плоскости), создаваемое точечным источником в трёхмерном волноводе мелкого моря, расположенным по координатам $x=y=0$, $z=z_s$ и имеющего частоту $f$, описывается трёхмерным уравнением Гельмгольца \cite{jensen}
            \begin{equation}\label{eq::3DH}
            \pa{\rho\pa{x,y,z}\nabla\cdot\pa{\frac{1}{\rho\pa{x,y,z}}\nabla} + K\pa{x,y,z}^2}p\pa{x,y,z}=-\delta\pa{x}\delta\pa{y}\delta\pa{z-z_s}
            \end{equation}
            где $\rho\pa{x,y,z}$ --- плотность среды. Коэффициент $K\pa{x,y,z}$ определяется двумя способами:
            \begin{enumerate}
                \item Без учёта затухания:
                \begin{equation}\label{eq::K1}
                  K\pa{x,y,z}=\frac{\omega}{c\pa{x,y,z}}\,,
                \end{equation}
                \item С учётом затухания:
                \begin{equation}\label{eq::K2}
                    K\pa{x,y,z}=\frac{\omega}{c\pa{x,y,z}}\pa{1+i\eta\beta}\,,
                \end{equation}
            \end{enumerate}
            где $c\pa{x,y,z}$ --- скорость звука, $\omega=2\pi f$ --- циклическая частота, $\beta$ --- коэффициент затухания, $\eta=\frac{1}{40\pi\log_{10}e}$.
            \newpage
            \par С использованием техники разделения переменных звуковое поле $p\pa{x,y,z}$ может быть представлено в виде модового разложения:
            \begin{equation}\label{eq::MD}
                p\pa{x,y,z}=\sum\limits_{j=1}^\infty A_j\pa{x,y}\varphi_j\pa{z},
            \end{equation}
            где $A_j\pa{x,y}$ --- модовые амплитуды, являющиеся решениями так называемого уравнения горизонтальной рефракции
            \begin{equation}\label{eq::HRE}
                \pa{\Delta+k_j^2\pa{x,y}}A_j\pa{x,y}=-\varphi_j\pa{z_s}\delta\pa{x}\delta\pa{y}\,.
            \end{equation}
            Модовые функции $\varphi_j\pa{z,x,y}$ и соответствующие им горизонтальные волновые числа $k_j\pa{x,y}$ находятся из решения акустической спектральной задачи \cite{jensen}.
        \subsubsection{Модовые параболические уравнения}
            \par Для получения параболических аппроксимаций уравнение горизонтальной рефракции \eqref{eq::HRE} представляется в виде
            \begin{equation}
                \pa{\frac{\partial}{\partial x}+i\sqrt{k_j^2\pa{x,y}+\frac{\partial^2}{\partial y^2}}}\pa{\frac{\partial}{\partial x}-i\sqrt{k_j^2\pa{x,y}+\frac{\partial^2}{\partial y^2}}}A_j\pa{x,y}=0\,,
            \end{equation}
            и выделяется его решение, состоящее из волн, распространяющихся в положительном направлении оси $x$
            \begin{equation}
                \pa{\frac{\partial}{\partial x}-i\sqrt{k_j^2\pa{x,y}+\frac{\partial^2}{\partial y^2}}}A_j\pa{x,y}=0\,.
            \end{equation}
            Вводя модовое опорное волновое число $k_{j,0}$ и выделяя главную осцилляцию из $A_j\pa{x,y}$
            \begin{equation}
                A_j\pa{x,y}=e^{k_{j,0}x}\mathcal{A}_j\pa{x,y}\,,\nonumber
            \end{equation}
            получим псевдодифференциальное модовое параболическое уравнение
            \begin{equation}\label{eq::PDMPE}
                \frac{\partial\mathcal{A}_j\pa{x,y}}{\partial x}=ik_{j,0}\pa{\sqrt{1+L_j}-1}\mathcal{A}_j\pa{x,y}\,,
            \end{equation}
            где 
            \begin{equation}
                k_{j,0}^2L_j=\frac{\partial^2}{\partial y^2}+k_j^2\pa{x,y}-k_{j,0}^2\,.\nonumber
            \end{equation}
        \subsubsection{Уравнение Гамильтона-Якоби}
            \par Предполагая, что волновые числа $k_j\pa{x,y}$ являются медленно изменяющейся функцией, решение уравнения \eqref{eq::HRE} с использованием лучевой теории распространения звука может быть выражено в виде
            \begin{equation}
                A_j\pa{x,y}=M_j\pa{x,y}e^{ik_{j,0}S_j\pa{x,y}}+o\pa{\nicefrac{1}{k_{j,0}}}\,,
            \end{equation}
            где функция $S_j\pa{x,y}$ называется эйконалом и может быть найдена из уравнения Гамильтона-Якоби
            \begin{equation}
                \pa{\frac{\partial S_j\pa{x,y}}{\partial x}}^2+\pa{\frac{\partial S_j\pa{x,y}}{\partial y}}^2=n_j\pa{x,y}\,,
            \end{equation}
            где $n_j\pa{x,y}\equiv \nicefrac{k_j\pa{x,y}}{k_{j,0}}$ --- индекс горизонтальной рефракции \cite{burridge}. Решение этого уравнения связано с решением так называемой Гамильновой системы
            \begin{equation}
                \begin{aligned}
                    \frac{dx_j\pa{l}}{dl}&=\frac{\xi_j\pa{l}}{n_j\pa{x,y}}\,,\qquad&\frac{d\xi_j\pa{l}}{dl}&=\frac{\partial n_j\pa{x,y}}{\partial x}\,,\\
                    \frac{dy_j\pa{l}}{dl}&=\frac{\eta_j\pa{l}}{n_j\pa{x,y}}\,,\qquad&\frac{d\eta_j\pa{l}}{dl}&=\frac{\partial n_j\pa{x,y}}{\partial y}\,,\\
                \end{aligned}
            \end{equation}
            где $l$ является натуральным параметром, обозначающим длину кривой вдоль траектории распространения луча, а $\xi,\eta$ сопряжённые переменные к $x,y$ --- момент. Из решения этой системы определяются траектории горизонтальных лучей распространения звука.
    \subsection{Неформальная постановка задачи}
        \par В рамках данной работы необходимо выполнить следующие задачи:
        \begin{enumerate}
            \item Исследовать методы моделирования трёхмерного звукового поля в сложных неоднородных океанических волноводах.
            \item Изучить методы решения МПУ с использованием условий прозрачной границы.
            \item Разработать комплекс программ для моделирования трёхмерных звуковых полей с учёта неоднородностей среды и возможностью визуализации.
        \end{enumerate}
        \subsubsection{Требования к программной реализации}
            \par На данный момент не существует программных комплексов позволяющих удобно вычислять и визуализировать звуковое поле. Наиболее популярным решением является использование различных программ для обработки входных данных, расчёта звукового поля и визуализации решения. Поэтому возникает задача разработки новой программы, позволяющий автоматизировать этот процесс. Основными требованиями к разрабатываемой программе являются:
            \begin{enumerate}
                \item Реализация численных схем решения модовых параболических уравнений.
                \item Возможность использования как готовых коэффициентов уравнения, так
и коэффициентов, вычисленных с помощью пакета Cambala \cite{cambala}, с указанием необхо­димых параметров среды: плотности среды, батиметрия, гидрология и др.
                \item Возможность расчёта и трассировки горизонтальных лучей распространения звука.
                \item Высокая скорость работы по сравнению с альтернативными методами моделирования.
                \item Возможность использования популярных форматов данных батиметрии и гидрологии.
                \item Наличие нескольких форматов вывода: текстовый, бинарный, изображения, файл фигур Matlab.
                \item Наличие пользовательского интерфейса для проведения моделирования, с возможностью задать параметры источника, приёмника, среды и проводимых вычислений.
                \item Возможность визуализации звукового поря и горизонтальных звуковых лучей с привязкой к конкретным географическим объектам.
            \end{enumerate}
    \subsection{Обзор существующих методов решения}
        \par На данный момент существует несколько программных продуктов позволяющих вычислять численное решение уравнения Гельмгольца \eqref{eq::3DH}. BELLHOP \cite{bellhop} и Traceo3D \cite{traceo}, основанные на методе суммирования Гауссовых пучков и лучевой теории распространения звука соответственно. Недостатком этих методов является использование геометроакустического приближения, которое является недостаточно точным при моделировании источников звука, имеющих частоту менее 1 кГц. Океанографический институт в Вудс-Хоуле \cite{whoi} и центральная школа Лиона \cite{lyon} имеют закрытые комплексы программ, основанные на решении трёхмерного параболического уравнения \cite{isakson14,lin12,shtrum16}, однако решение таких уравнений требует запредельных затрат памяти и времени, расчёт самых простых задач занимает не менее суток.
        \par Для визуализации вычислений чаще всего используются специализированные пакеты,например Gnuplot \cite{gnuplot}, Matlab \cite{matlab}, Wolfram Mathematica \cite{wolfram}. Такие программы предлагают широкие возможности для визуализации, однако усложняют процесс работы, добавляя в него дополнительный шаг.
    \subsection{План работ}
        \begin{enumerate}
            \item Реализация численных методов решения модовых параболических уравнений с использованием условий прозрачной границы и учётом неоднородностей среды.
            \item Реализация расчёта горизонтальных лучей распространения звука.
            \item Добавление возможности использования различных форматов ввода и вывода данных.
            \item Реализация пользовательского интерфейса.
        \end{enumerate}
    \subsection{Математические методы}
        \subsubsection{Математическая постановка задачи}
            \par Необходимо в области $\Omega=\left\{\pa{x,y,z}\big|0\leqslant x\leqslant x_1,y_0\leqslant y\leqslant y_1,0\leqslant z\leqslant z_b\right\}$ найти звуковое поле $p\pa{x,y,z}$ точечного источника, являющееся решением уравнения Гельмгольца
            \begin{multline}
                \frac{\partial^2p\pa{x,y,z}}{\partial x^2}+\frac{\partial^2p\pa{x,y,z}}{\partial y^2}+\rho\pa{z}\frac{\partial}{\partial z}\pa{\frac{1}{\rho\pa{z}}\frac{\partial p\pa{x,y,z}}{\partial z}}+\\
                K^2\pa{z,x}p\pa{x,y,z}=-\delta\pa{x}\delta\pa{y}\delta\pa{z-z_s}\,,
            \end{multline}
            в форме модового разложения
            \begin{equation}
                p\pa{x,y,z}=\sum\limits_{j=1}^Je^{k_{j,0}x}\mathcal{A}_j\pa{x,y}\varphi_j\pa{z,x,y}\,.
            \end{equation}
            наблюдаемого приёмником, расположенным на глубине $z_r$. Модовые амплитуды $\mathcal{A}_j\pa{x,y}$ являются решениями соответствующих задач Коши
            \begin{equation}\label{eq::WAMPE_problem}
                \begin{dcases}
                    \frac{\partial\mathcal{A}_j\pa{x,y}}{\partial x}=ik_{j,0}\pa{\sqrt{1+L_j}-1}\mathcal{A}_j\pa{x,y}\,,\\
                    \mathcal{A}_j\pa{0,y}=\mathcal{A}_{j,0}\pa{y}\,.
                \end{dcases}
            \end{equation}
            При этом считается, что модововые функции $\varphi_j\pa{z,x,y}$ и соответствующие им волновые числа $k_j\pa{x,y}$ заранее известны из решения спектральной задачи.
        \subsubsection{Аппроксимация Паде}
            \par Пусть есть некоторая функция $F\pa{\lambda}$ тогда её аппроксимация Паде может быть записана в виде
            \begin{equation}
                F\pa{\lambda}\approx\mathcal{R}\pa{F,l,m}\pa{\lambda}\equiv\frac{P_{l,m}^F\pa{\lambda}}{Q_{l,m}^F\pa{\lambda}}\,,
            \end{equation}
            где $P_{l,m}^F\pa{\lambda},Q_{l,m}^F\pa{\lambda}$ обозначают многочлены порядка $l$ и $m$ соответственно \cite{jensen}. Коэффициенты многочленов могут быть получены приравниванием рациональной функции $\nicefrac{P_{l,m}^F\pa{\lambda}}{Q_{l,m}^F\pa{\lambda}}$ к разложению в ряд Тейлора функции $F\pa{\lambda}$, содержащей $l+m+1$ членов.
        \subsubsection{Аппроксимация оператора квадратного корня}
            \par Для решения уравнения \eqref{eq::PDMPE} необходимо выполнить линеаризацию оператора квадратного корня с использованием аппроксимации Паде
            \begin{equation}\label{eq::pade_mpe}
                \frac{\partial\mathcal{A}_j\pa{x,y}}{\partial x}=ik_{j,0}\pa{\frac{P_{l,m}^F\pa{L_j}}{Q_{l,m}^F\pa{L_j}}-1}\mathcal{A}_j\pa{x,y}\,,
            \end{equation}
            где $F\pa{\cdot}=\sqrt{\cdot}$. Используя дискретизацию Крэнка-Николсон \cite{crank} на равномерной сетке $x=nh$, $\mathcal{A}_j^n\sim\mathcal{A}_j\pa{x_n,y}$, уравнение \eqref{eq::pade_mpe} в положительном направлении оси $x$ можно записать виде
            \begin{equation}\label{eq::CNMPE}
                D_h^+\mathcal{A}_j^n=ik_{j,0}\pa{\frac{P_{l,m}^F\pa{L_j}}{Q_{l,m}^F\pa{L_j}}-1}\mathcal{A}_j^{\nicefrac{n+1}{2}}\,,
            \end{equation}
            где 
            \begin{align*}
                D_h^+=\mathcal{A}_j=\frac{\mathcal{A}_j^{n+1}-\mathcal{A}_j^n}{h}\,,&&\mathcal{A}_j^{\nicefrac{n+1}{2}}=\frac{\mathcal{A}_j^{n+1}+\mathcal{A}_j^n}{2}\,.
            \end{align*}
            Уравнение \eqref{eq::CNMPE} может быть преобразовано к виду
            \begin{equation}
                \mathcal{A}_j^{n+1}=\frac{U\pa{L_j}}{W\pa{L_j}}\mathcal{A}_j^n\,,
            \end{equation}
            где
            \begin{align*}
                U\pa{L_j}&=\pa{2-ik_{j,0}h}Q_{l,m}^F\pa{L_j}+ik_{j,0}hP_{l,m}^F\pa{L_j}\,,\\
                W\pa{L_j}&=\pa{2+ik_{j,0}h}Q_{l,m}^F\pa{L_j}-ik_{j,0}hP_{l,m}^F\pa{L_j}\,.
            \end{align*}
            многочлены степени $p=\max\pa{l,m}$. Разложив отношение $\nicefrac{U\pa{L_j}}{W\pa{L_j}}$ на простые дроби получим
            \begin{equation}\label{eq::dPDMPE}
                \mathcal{A}_j^{n+1}=\pa{1+\sum\limits_{i=1}^p\frac{a_{l,m}^iL_j}{1+b_{l,m}^iL_j}}\mathcal{A}_j^n\,.
            \end{equation}
        \subsubsection{Split-step Pad\'e}
            \par Другой подход к решению \eqref{eq::PDMPE} был предложен Коллинзом \cite{collins}. В его основе лежит смена порядка дискретизации и применения аппроксимации Паде. При достаточно маленьком интервале $\Delta x=h$ уравнение \eqref{eq::PDMPE} может быть формально решено в виде
            \begin{equation}
                \mathcal{A}_j^{n+1}=e^{ik_{j,0}h\pa{\sqrt{1+L}-1}}\mathcal{A}_j^n\,.
            \end{equation}
            Затем, применяя аппроксимацию Паде для экспоненты в виде разложения на простые дроби
            \begin{equation}\label{eq::SSPADE}
                e^{ik_{j,0}h\pa{\sqrt{1+L}-1}}\approx\frac{\tilde{U}\pa{L_j}}{\tilde{W}\pa{L_j}}= 1+\sum\limits_{i=1}^p\frac{\tilde{a}_{l,m}^iL_j}{1+\tilde{b}_{l,m}^iL_j}\,,
            \end{equation}
            получим
            \begin{equation}
                \mathcal{A}_j^{n+1}=\pa{1+\sum\limits_{i=1}^p\frac{\tilde{a}_{l,m}^iL_j}{1+\tilde{b}_{l,m}^iL_j}}\mathcal{A}_j^n\,.
            \end{equation}
        \subsubsection{Дискретизация оператора $L_j$}
            \par Для дискретизации уравнения \eqref{eq::dPDMPE} на равномерной сетке $y_q=y_0+q\delta$ с шагом $\Delta y=\delta$ используется стандартная конечно-разностная схема
            \begin{equation}
                D_\delta^2\mathcal{A}_j^{n+1}=\frac{\mathcal{A}_j^{n+1,q+1}-2\mathcal{A}_j^{n+1,q}+\mathcal{A}_j^{n+1,q-1}}{\delta^2}\,,\quad q\in\mathbb{N}\,,
            \end{equation}
            где $\mathcal{A}_j^{n,q}\sim\mathcal{A}_j\pa{x_n,y_q}$. Таким образом, уравнение \eqref{eq::dPDMPE} преобразуется к виду
            \begin{equation}\label{eq::ddPDMPE}
                \mathcal{A}_j^{n+1,q}=\pa{1+\sum\limits_{i=1}^p\frac{a_{l,m}^iL_j^\delta}{1+b_{l,m}^iL_j^\delta}}\mathcal{A}_j^{n,q}\,,\quad q\in\mathbb{N}\,,
            \end{equation}
            где $k_{j,0}^2L_j^\delta=D_\delta^2+k_j^2+k_{j,0}^2$. Вводятся вспомогательные функции $\mathcal{B}_{j,i}^{n+1,q},i=\overline{1,p}$ такие, что 
            \begin{align}
                \pa{1+b_{l,m}^iL_j^\delta}\mathcal{B}_{j,i}^{n+1,q}&=a_{l,m}^iL_j^\delta\mathcal{A}_j^{n,q}\,,&&i=\overline{1,p-1}\,,\\
                \pa{1+b_{l,m}^pL_j^\delta}\mathcal{B}_{j,p}^{n+1,q}&=\pa{1+\pa{a_{l,m}^p+b_{l,m}^p}L_j^\delta}\mathcal{A}_j^{n,q}\,,&&i=p\,.
            \end{align}
            Коэффициенты $\mathcal{B}_{j,i}^{n+1,q}$ могут быть легко найдены обращением трёхдиагональных матриц $\pa{1+b_{l,m}^iL_j^\delta}$, что  приводит к финальному результату
            \begin{equation}
                \mathcal{A}_j^{n+1,q}=\sum\limits_{i=1}^p\mathcal{B}_{j,i}^{n+1,q}\,,\quad q\in\mathbb{N}\,.
            \end{equation}
            \par Для использования дискретизации оператора $L_j$ с использованием SSP применяется следующий подход, предложенный Коллинзом \cite{collins}. Запишем формально 
            \begin{equation}
                \mathcal{A}_j^n\pa{y_{q+1}}=e^{\delta\partial_y}\mathcal{A}_j^n\pa{y_q}=e^{\delta k_{j,0}\sqrt{L_j-\pa{\nicefrac{k_j}{k_{j,0}}-1}}}\mathcal{A}_j^n\pa{y_q}\,,
            \end{equation}
            и получим следующее выражение
            \begin{multline}
                L_\delta=-\frac{e^{\tau\sqrt{L_j-\pa{\nicefrac{k_j}{k_{j,0}}-1}}}+2+e^{-\tau\sqrt{L_j-\pa{\nicefrac{k_j}{k_{j,0}}-1}}}}{\tau^2}=\\
                -2\frac{\cosh\pa{\tau\sqrt{L_j-\pa{\nicefrac{k_j}{k_{j,0}}-1}}}-1}{\tau^2}\,,\quad\tau=\delta k_{j,0}\,.
            \end{multline}
            Решение этого уравнения относительно $L_j$ позволяет получить функцию, зависящую от $L_j^\delta$
            \begin{equation}
                L_j=\Gamma_j\pa{L_j^\delta}=\tau^{-2}\log^2\pa{1-\frac{\tau^2}{2}L_j^\delta+\sqrt{\pa{1-\frac{\tau^2}{2}L_j^\delta}^2-1}}+\frac{k_j}{k_{j,0}}-1\,.
            \end{equation}
            Таким образом 
            \begin{equation}
                \mathcal{A}_j^{n+1,q}=e^{ik_{j,0}h\pa{\sqrt{1+\Gamma\pa{L_j^\delta}}-1}}\mathcal{A}_j^{n,q}\,.
            \end{equation}
            Используя аппроксимацию Паде аналогично \eqref{eq::SSPADE} получим
            \begin{equation}
                \mathcal{A}_j^{n+1,q}=\pa{1+\sum\limits_{i=1}^p\frac{\tilde{a}_{l,m}^iL_j^\delta}{1+\tilde{b}_{l,m}^iL_j^\delta}}\mathcal{A}_j^{n,q}\,,\quad q\in\mathbb{N}\,,
            \end{equation}
            Полученное дискретное уравнение может быть решено аналогично \eqref{eq::ddPDMPE}.
\end{document}