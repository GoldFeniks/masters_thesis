\documentclass[../document.tex]{subfiles}

\begin{document}
    \subsection{Описание предметной области}
        \subsubsection{Оценка влияния антропогенных шумов}
            \par Моделирование трёхмерных звуковых полей применяется во многих областях исследования и освоения океана, требующих учёта множества параметров сложных неоднородных океанических волноводов.
            \par С развитием промышленности все больше расширяется хозяйственная деятельность человека, связанная с добычей нефти, газа и разнообразных биоресурсов в акватории мирового океана, в результате которой создаётся огромное количество антропогенных шумов, которые негативно сказываются на морской фауне \cite{noise1, noise2}. Таким образом, возникает задача оценивания и минимизации шумового загрязнения и его воздействия на мировой океан. Существуют два подхода к решению этой задачи: проведение замеров и моделирование. В первом случае проводится некоторое количество измерений плотности звука в воде, по которым строится интерполяционная картина звукового поля. Недостатком такого метода является дороговизна и сложность проведения замеров, поэтому такие данные чаще всего используются для корректировки и проверки точности модельных данных. В свою очередь моделирование требует данных о положении и характеристиках источника звука, а также данных о свойствах среды: батиметрии, гидрологии и структуре дна. Основным преимуществом моделирования является возможность вычисления звукового поля как и уже существующих источников, так и планируемых, что позволяет заранее минимизировать влияние человека на океан. Минусом такого метода является необходимость сбора и обработки изменяющихся данных о состоянии среды, что само по себе является сложной задачей.
        \subsubsection{Акустическая навигация}
            \par С каждым годом хозяйственная деятельность человека всё больше осуществляется с использованием автономных подводных аппаратов, требующих наличия стабильных систем навигации и связи, основанных на распространении звука, так как электромагнитное излучение неприменимо в подобных условиях \cite{navigation19,navigation20}. При разработке систем подводной акустической навигации возникает задача определения зон уверенного приёма и поиска взаимного расположения источников звукового сигнала таким образом, чтобы минимизировать зоны акустической тени. Также существует задача расчёта траекторий распространения звука на несущих частотах сигналов от 300 до 500 Гц, с целью определения искривления звуковых лучей для вычисления задержки звукового сигнала при осуществлении подводной навигации.
        \subsubsection{Модовое представление звукового поля точечного источника}
            \par Звуковое поле $p\pa{x,y,z}$ (где $z$ обозначает глубину, а  $x,y$ --- координаты горизонтальной плоскости), создаваемое точечным источником в трёхмерном волноводе мелкого моря, расположенным по координатам $x=y=0$, $z=z_s$ и имеющего частоту $f$, описывается трёхмерным уравнением Гельмгольца \cite{jensen}
            \begin{equation}\label{eq::3DH}
            \pa{\rho\pa{x,y,z}\nabla\cdot\pa{\frac{1}{\rho\pa{x,y,z}}\nabla} + K\pa{x,y,z}^2}p\pa{x,y,z}=-\delta\pa{x}\delta\pa{y}\delta\pa{z-z_s}
            \end{equation}
            где $\rho\pa{x,y,z}$ --- плотность среды. Коэффициент $K\pa{x,y,z}$ определяется двумя способами:
            \begin{enumerate}
                \item Без учёта затухания:
                \begin{equation}\label{eq::K1}
                  K\pa{x,y,z}=\frac{\omega}{c\pa{x,y,z}}\,,
                \end{equation}
                \item С учётом затухания:
                \begin{equation}\label{eq::K2}
                    K\pa{x,y,z}=\frac{\omega}{c\pa{x,y,z}}\pa{1+i\eta\beta}\,,
                \end{equation}
            \end{enumerate}
            где $c\pa{x,y,z}$ --- скорость звука, $\omega=2\pi f$ --- циклическая частота, $\beta$ --- коэффициент затухания, $\eta=\frac{1}{40\pi\log_{10}e}$.
            \newpage
            \par С использованием техники разделения переменных звуковое поле $p\pa{x,y,z}$ может быть представлено в виде модового разложения:
            \begin{equation}\label{eq::MD}
                p\pa{x,y,z}=\sum\limits_{j=1}^\infty A_j\pa{x,y}\varphi_j\pa{z},
            \end{equation}
            где $A_j\pa{x,y}$ --- модовые амплитуды, являющиеся решениями так называемого уравнения горизонтальной рефракции
            \begin{equation}\label{eq::HRE}
                \pa{\Delta+k_j^2\pa{x,y}}A_j\pa{x,y}=-\varphi_j\pa{z_s}\delta\pa{x}\delta\pa{y}\,.
            \end{equation}
            Модовые функции $\varphi_j\pa{z,x,y}$ и соответствующие им горизонтальные волновые числа $k_j\pa{x,y}$ находятся из решения акустической спектральной задачи \cite{jensen}.
        \subsubsection{Модовые параболические уравнения}
            \par Для получения параболических аппроксимаций уравнение горизонтальной рефракции \eqref{eq::HRE} представляется в виде
            \begin{equation}
                \pa{\frac{\partial}{\partial x}+i\sqrt{k_j^2\pa{x,y}+\frac{\partial^2}{\partial y^2}}}\pa{\frac{\partial}{\partial x}-i\sqrt{k_j^2\pa{x,y}+\frac{\partial^2}{\partial y^2}}}A_j\pa{x,y}=0\,,
            \end{equation}
            и выделяется его решение, состоящее из волн, распространяющихся в положительном направлении оси $x$
            \begin{equation}
                \pa{\frac{\partial}{\partial x}-i\sqrt{k_j^2\pa{x,y}+\frac{\partial^2}{\partial y^2}}}A_j\pa{x,y}=0\,.
            \end{equation}
            Вводя модовое опорное волновое число $k_{j,0}$ и выделяя главную осцилляцию из $A_j\pa{x,y}$
            \begin{equation}
                A_j\pa{x,y}=e^{k_{j,0}x}\mathcal{A}_j\pa{x,y}\,,\nonumber
            \end{equation}
            получим псевдодифференциальное модовое параболическое уравнение
            \begin{equation}\label{eq::PDMPE}
                \frac{\partial\mathcal{A}_j\pa{x,y}}{\partial x}=ik_{j,0}\pa{\sqrt{1+L_j}-1}\mathcal{A}_j\pa{x,y}\,,
            \end{equation}
            где 
            \begin{equation}
                k_{j,0}^2L_j=\frac{\partial^2}{\partial y^2}+k_j^2\pa{x,y}-k_{j,0}^2\,.\nonumber
            \end{equation}
        \subsubsection{Уравнение Гамильтона-Якоби}
            \par Предполагая, что волновые числа $k_j\pa{x,y}$ являются медленно изменяющейся функцией, решение уравнения \eqref{eq::HRE} с использованием лучевой теории распространения звука может быть выражено в виде
            \begin{equation}
                A_j\pa{x,y}=M_j\pa{x,y}e^{ik_{j,0}S_j\pa{x,y}}+o\pa{\nicefrac{1}{k_{j,0}}}\,,
            \end{equation}
            где функция $S_j\pa{x,y}$ называется эйконалом и может быть найдена из уравнения Гамильтона-Якоби
            \begin{equation}
                \pa{\frac{\partial S_j\pa{x,y}}{\partial x}}^2+\pa{\frac{\partial S_j\pa{x,y}}{\partial y}}^2=n_j\pa{x,y}\,,
            \end{equation}
            где $n_j\pa{x,y}\equiv \nicefrac{k_j\pa{x,y}}{k_{j,0}}$ --- индекс горизонтальной рефракции \cite{burridge}. Решение этого уравнения связано с решением так называемой Гамильновой системы
            \begin{equation}
                \begin{aligned}
                    \frac{dx_j\pa{l}}{dl}&=\frac{\xi_j\pa{l}}{n_j\pa{x,y}}\,,\qquad&\frac{d\xi_j\pa{l}}{dl}&=\frac{\partial n_j\pa{x,y}}{\partial x}\,,\\
                    \frac{dy_j\pa{l}}{dl}&=\frac{\eta_j\pa{l}}{n_j\pa{x,y}}\,,\qquad&\frac{d\eta_j\pa{l}}{dl}&=\frac{\partial n_j\pa{x,y}}{\partial y}\,,\\
                \end{aligned}
            \end{equation}
            где $l$ является натуральным параметром, обозначающим длину кривой вдоль траектории распространения луча, а $\xi,\eta$ сопряжённые переменные к $x,y$ --- момент. Из решения этой системы определяются траектории горизонтальных лучей распространения звука.
    \subsection{Неформальная постановка задачи}
        \par В рамках данной работы необходимо выполнить следующие задачи:
        \begin{enumerate}
            \item Исследовать методы моделирования трёхмерного звукового поля в сложных неоднородных океанических волноводах.
            \item Изучить методы решения МПУ с использованием условий прозрачной границы.
            \item Разработать комплекс программ для моделирования трёхмерных звуковых полей с учёта неоднородностей среды и возможностью визуализации.
        \end{enumerate}
        \subsubsection{Требования к программной реализации}
            \par На данный момент не существует программных комплексов позволяющих удобно вычислять и визуализировать звуковое поле. Наиболее популярным решением является использование различных программ для обработки входных данных, расчёта звукового поля и визуализации решения. Поэтому возникает задача разработки новой программы, позволяющий автоматизировать этот процесс. Основными требованиями к разрабатываемой программе являются:
            \begin{enumerate}
                \item Реализация численных схем решения модовых параболических уравнений.
                \item Возможность использования как готовых коэффициентов уравнения, так
и коэффициентов, вычисленных с помощью пакета Cambala \cite{cambala}, с указанием необхо­димых параметров среды: плотности среды, батиметрия, гидрология и др.
                \item Возможность расчёта и трассировки горизонтальных лучей распространения звука.
                \item Высокая скорость работы по сравнению с альтернативными методами моделирования.
                \item Возможность использования популярных форматов данных батиметрии и гидрологии.
                \item Наличие нескольких форматов вывода: текстовый, бинарный, изображения, файл фигур Matlab.
                \item Наличие пользовательского интерфейса для проведения моделирования, с возможностью задать параметры источника, приёмника, среды и проводимых вычислений.
                \item Возможность визуализации звукового поря и горизонтальных звуковых лучей с привязкой к конкретным географическим объектам.
            \end{enumerate}
    \subsection{Обзор существующих методов решения}
        \par На данный момент существует несколько программных продуктов позволяющих вычислять численное решение уравнения Гельмгольца \eqref{eq::3DH}. BELLHOP \cite{bellhop} и Traceo3D \cite{traceo}, основанные на методе суммирования Гауссовых пучков и лучевой теории распространения звука соответственно. Недостатком этих методов является использование геометроакустического приближения, которое является недостаточно точным при моделировании источников звука, имеющих частоту менее 1 кГц. Океанографический институт в Вудс-Хоуле \cite{whoi} и центральная школа Лиона \cite{lyon} имеют закрытые комплексы программ, основанные на решении трёхмерного параболического уравнения \cite{isakson14,lin12,shtrum16}, однако решение таких уравнений требует запредельных затрат памяти и времени, расчёт самых простых задач занимает не менее суток.
        \paragraph{Аппроксимация оператора квадратного корня}
        \par Для решения уравнения \eqref{eq::PDMPE} необходимо выполнить линеаризацию оператора квадратного корня. Для этого используется аппроксимация Паде \cite{jensen}
        \begin{equation}
            \sqrt{1+L_j}=1+\sum\limits_{i=1}^m\frac{a_{i,m}L_j}{1+b_{i,m}L_j}+O\pa{L_j^{2m+1}}\,,
        \end{equation}
        где $m$ --- количество членов аппроксимации, а коэффициенты $a_{i,m}$ и $b_{i,m}$ вычисляются по формулам
        \begin{align}
            a_{i,m}&=\frac{2}{2m+1}\sin^2\pa{\frac{i\pi}{2m+1}}\,,\\
            b_{i,m}&=\cos^2\pa{\frac{i\pi}{2m+1}}\,.
        \end{align}
        Используя такую аппроксимацию из уравнения \eqref{eq::PDMPE} получим модовое параболическое уравнение
        \begin{equation}
            \frac{\partial\mathcal{A}_j\pa{x,y}}{\partial x}=ik_{j,0}\sum\limits_{i=1}^m\frac{a_{i,m}L_j}{1+b_{i,m}L_j}\mathcal{A}_j\pa{x,y}\,.
        \end{equation}
\end{document}