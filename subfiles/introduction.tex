\documentclass[../document.tex]{subfiles}

\begin{document}
	\section*{Введение}
	    \par Моделирование трёхмерных звуковых полей применяется во многих областях исследования и освоения океана, требующих учёта множества параметров сложных неоднородных океанических волноводов, например, оценка влияния антропогенных шумов и акустическая навигация \cite{noise1, noise2, navigation19, navigation20}. В первую очередь для таких исследований требуется возможность эффективного и точного вычисления звукового поля в пространстве, а также импульсов звуковых сигналов. На данный момент существует достаточно много программ и библиотек, имеющих требуемый функционал, однако зачастую такие программы решают лишь узкоспециализированные задачи. Также, многие из них основаны на недостаточно точных методах, например, узкоугольные параболические уравнения, или недостаточно эффективных, например, трёхмерные параболические уравнения \cite{isakson14,lin12,shtrum16,whoi,lyon} и метод суммирования Гауссовых пучков \cite{bellhop,traceo}. Таким образом поставлена цель разработать программный комплекс, позволяющий решать различные задачи моделирования, основываясь на решении широкоугольного параболического уравнения, обладающего достаточной эффективностью и точностью.
	    \par Для достижения поставленной цели необходимо реализовать следующий функционал:
	    	\begin{itemize}
	    		\item Вычисление временного ряда импульса звукового сигнала в произвольных точках волновода с указанием сигнала в одной из них или функции источника.
	    		\item Расчёт интегральных характеристик (\acrshort{sel}).
	    		\item Split-step Pad\'e метод решения широкоугольного параболического уравнения \cite{collins}.
                \item Использование \acrshort{pml} граничных условий \cite{berenger, levy, lu}.
	    		\item Поддержка аппроксимации Паде произвольного порядка.
	    		\item Усовершенствованный процесс ввода и вывода данных.
	    		\item Проверка входных данных.    		
	    	\end{itemize}
	    \par Рассматриваемые в работе широкоугольные модовые параболические уравнения известны уже достаточно давно, однако не получили широкого распространения. Несмотря на это, использование таких уравнений является перспективным, так как в сравнении с узкоугольными параболическими уравнениями они позволяют получать более точные решения, при этом требуя лишь незначительно больше вычислений. Так, самым трудоёмким этапом является решение акустической спектральной задачи, которое может быть вычислено заранее для изучаемой области, что значительно ускоряет процесс моделирования, так как само решение уравнения занимает лишь несколько минут. В работе также изучено применение лучевых стартеров, которые являются более подходящими для решения широкоугольных параболически уравнений в сравнении с традиционно используемыми начальными условиями Гаусса и Грина.
	    \par Разрабатываемый в рамках данной работы программный продукт упрощает процесс моделирования, абстрагируя исследователя от необходимости вручную редактировать программный код, чтобы задать свойства среды. Все необходимые параметры могут быть заданы в конфигурационном файле, в том числе данные батиметрии и гидрологии. Исходный код программы размещён в открытом доступе, что не только позволяет использовать его в качестве сторонней библиотеки, но и предоставляет возможность его редактирования под нужды пользователя при необходимости. Ещё одним плюсом такого подхода является возможность сторонним разработчикам предлагать изменения исходный код программы, что позволяет программного продукты быстрее развиваться и больше соответствовать требованиям пользователей.
	    \par В первую очередь работа фокусируется на исследовании широкоугольных параболических уравнений, содержащих аппроксимацию Паде произвольного порядка, и методов их решения, рассматриваются различные варианты дискретизаций и граничных условий. Далее изучается использование двух видов лучевых стартеров и производится их сравнение с традиционными начальными условия Гаусса и Грина \cite{jensen}. Рассматривается задаче вычисления импульса звукового сигнала по функции источника или известному сигналу в одном из приёмников.
	    \par Первая глава посвящена исследованию предметной области, дано подробное описание рассматриваемых уравнений, методов, математических терминов, использованных в работе, поставлена математическая постановка задачи, проведёт анализ существующих методов решения, выявлены их недостатки и способы борьбы с ними.
	    \par Во второй главе подробно рассматриваются широкоугольные параболические уравнения, способы и тонкости их решения, производится теоретическое сравнение с существующими методами.
	    \par Третья глава содержит описание разработанного продукта и результаты набора вычислительных экспериментов, направленных на изучение эффективности и точности использованных методов.
\end{document}