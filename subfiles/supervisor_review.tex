\documentclass[../document.tex]{subfiles}

\begin{document}
    \begin{supervisorreview}[Тыщенко Андрея Геннадьевича]{отлично}{97.97}{02.07}
        \par Выпускная квалификационная работа магистра (ВКР) Тыщенко А.Г. посвящена разработке математических методов и комплексов программ для решения актуальных практических задач акустики океана, в которых возникает необходимость расчёта акустических полей, формируемых точечными источниками импульсных и тональных сигналов в реальных волноводах мелкого моря. К числу приложений, в которых может быть использован комплекс программ, можно отнести задачи акустического мониторинга и задачи организации систем акустической навигации на обширных акваториях. Задачи мониторинга антропогенных акустических шумов на шельфе исключительно актуальны в настоящее время ввиду необходимости сохранения и защиты морской фауны при освоении ресурсов Мирового Океана (в частности, при разведке и разработке месторождений углеводородов). Системы акустической навигации, в свою очередь, являются критически важным элементом инфраструктуры, необходимой для обеспечения функционирования автономных необитаемых подводных аппаратов на большом удалении от центров управления их миссиями. Для обоих указанных классов практических задач важно уметь моделировать распространение акустических сигналов в сложных неоднородных океанических волноводах с учётом трёхмерного характера распространения и звука, в частности, с учётом горизонтальной рефракции звука на неоднородностях рельефа дна и поля скорости звука. Данное моделирование сводится к решению уравнений, описывающих распространение звука (например, уравнения Гельмгольца) для обширных акваторий площадью сотни и тысячи квадратных километров. Соответствующие расчёты не могут быть выполнены существующими методами за обозримое время, и потому основной задачей, поставленной перед А.Г. Тыщенко, была разработка нового комплекса программ, основанного на использовании последних достижений вычислительной акустики океана и прикладной математики, которых бы позволил справиться с вычислением звуковых полей в перечисленных случаях. 
        \par Автор ВКР, А.Г. Тыщенко, блестяще справился с поставленной задачей, и итогам его работы стал комплекс прикладных программ \guillemotleft AMPLE\guillemotright, который позволяет выполнять расчёт акустических полей тональных источников, а также рассчитывать временные ряды импульсных сигналов в точках приёма. Данный комплекс программ основан на численном решении широкоугольных и псевдодифференциальных модовых параболических уравнений с помощью метода конечных разностей и метода SSP. Отдельные элементы методики численного решения разработаны лично Тыщенко А.Г., а программный код спроектирован и реализован единолично с использованием языка программирования С++. Комплекс программ \guillemotleft AMPLE\guillemotright\ был успешно апробирован на серии модельных тестовых задач, где звуковые поля могут быть вычислены и другими методами для выполнения сравнительного анализа точности расчётов. Тестирование показало, что, благодаря использованию наиболее эффективных вычислительных методов, комплекс программ \guillemotleft AMPLE\guillemotright\ обеспечивает существенно более высокую эффективность расчётов, чем программы, основанные на методе трёхмерных параболических уравнений (основного средства, используемого для решения задач расчёта звуковых полей в акустике океана). В настоящее время комплекс программ уже используется сотрудниками лаб. 2/4 ТОИ ДВО РАН при проведении мониторинга антропогенных акустических шумов на шельфе о. Сахалин. 
        Результаты работы Тыщенко А.Г. были доложены на крупных международных научных мероприятиях, включая конференции PACON, UACE, Days on Diffraction. Большая часть докладов была сделано лично А.Г. Тыщенко. Доклады вызвали живой интерес со стороны ученых-акустиков из разных стран. По результатам работ опубликована статья в \guillemotleft Journal of Sound and Vibration\guillemotright\ (Q1 WoS), а также направлена статья в \guillemotleft Акустический журнал\guillemotright\ (в настоящее время она принята к печати и выйдет в конце 2021 г.).
        \par Считаю, что работа Тыщенко А.Г. полностью соответствует всем требования, предъявляемым к ВКР магистра и заслуживает оценки \guillemotleft отлично\guillemotright. А.Г. Тыщенко заслуживает присуждения степени \guillemotleft магистр\guillemotright\ по направлению подготовки 09.04.01 -- \guillemotleft Информатика и вычислительная техника\guillemotright.
    \end{supervisorreview}
\end{document}