\documentclass[../document.tex]{subfiles}

\begin{document}
    \begin{thesistask}[Тыщенко Андрею Геннадьевичу]{24.12.20}{124-01-03-18}{05.07.21}{23.12.20}
        \taskitem
        \uline{
            Разработка программного кода на языке С++ с возможностью использования (ввода) данных натур\-ных гидрологических и батиметрических измерений, а также данных о структуре слоёв морского дна  (текстовые файлы), обеспечивающего расчёт звуковых полей на одном или нескольких гори\-зонтах. Результаты расчётов должны выводиться с текстовые и/или бинарные файлы. Должна быть реализована возможность расчёта тональных звуковых полей, временных рядов импульсных сигналов в точках приёма, а также уровней озвучивания области (SEL). Расчёты для одной частоты в диапазоне от 10 до 500 Гц должны выполняться не более, чем за 1 час для области размером 500 м (глубина) х 10 км (х) х 5 км (у).\hfill\null
        }
        \taskitem
        \uline{
            Метод SSP и особенности его применения в областях с прозрачными границами. Методика расчёта звукового поля точечного источника в волноводе мелкого моря с использованием данных о бати\-метрии, гидрологии и рельефе дна. Комплекс программ на языке C++, реализующий разработан\-ную методику. Тестирование комплекса программ на серии модельных задач, в которых рассматри\-вается распространение звука в клиновидном прибрежном волноводе, волноводе мелкого моря с подводным каньоном и волноводе мелкого моря с реальной батиметрией и гидрологией. Рабочая документация к комплексу программ.\hfill\null
        }
        \taskitem
        \uline{
            Jensen F.B., Kuperman W.A., Porter M.B., Schmidt H. Computational Ocean Acoustics. New-York: Springer, 2012. 794 p.\hfill\null\\
            Arnold A., Ehrhardt M. Discrete transparent boundary conditions for wide angle parabolic equations in underwater acoustics // Journal of Computational Physics, 1998, V. 145, No. 2, P. 611--638.\hfill\null\\
            Petrov P.S., Antoine X. Pseudodifferential adiabatic mode parabolic equations in curvilinear coordinates and their numerical solution//Journal of Computational Physics, 2020, V.410, art. no. 109392.\hfill\null\\
            Lin Y.T., Duda T.F., Newhall A.E. Three-dimensional sound propagation models using the parabolic\-equation approximation and the split-step Fourier method // J. Comput. Acoust. 2013. V. 21. № 1. P. 1250018.\hfill\null\\
            Petrov P.S., Ehrhardt M., Tyshchenko A.G., Petrov P.N. Wide-angle mode parabolic equations for the modelling of horizontal refraction in underwater acoustics and their numerical solution on unbounded domains//Journal of Sound and Vibration, 2020, V. 484, art. no. 115526.\hfill\null
        }
    \end{thesistask}
\end{document}